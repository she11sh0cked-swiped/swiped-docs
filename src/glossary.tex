\newglossaryentry{FrontEnd}
{   name={Front-End},
    description={Das Front-End bezeichnet den für den Endbenutzer sichtbaren Teil einer Anwendung. (Vergleich Back-End)}
}

\newglossaryentry{BackEnd}
{   name={Back-End},
    description={Das Back-End bezeichnet den für den Endbenutzer nicht sichtbaren Teil einer Anwendung. (Vergleich Front-End)}
}

\newglossaryentry{React}
{   name={React},
    description={React ist eine Front-End JavaScript-Bibliothek zum Erstellen von Benutzeroberflächen}
}

\newglossaryentry{NodeJS}
{   name={NodeJS},
    description={NodeJS ist eine Back-End JavaScript-Laufzeitumgebung die es ermöglicht JavaScript-Code außerhalb eines Webbrowsers auszuführen}
}

\newglossaryentry{MongoDB}
{   name={MongoDB},
    description={MongoDB ist ein dokumenten-orientiertes Datenbank Programm}
}

\newglossaryentry{NginX}
{   name={NginX},
    description={NginX ist ein Webserver der auch als Reverse-Proxy, Load-Balancer, Mail-Proxy und HTTP cache verwendet werden kann}
}

\newglossaryentry{Express}
{   name={Express},
    description={Express ist eine Back-End JavaScript-Bibliothek zum Erstellen von Webanwendungen und APIs unter NodeJS}
}

\newglossaryentry{OpenSource}
{   name={Open-Source},
    description={Das Open-Source-Modell ist ein dezentrales Softwareentwicklungsmodell, das eine offene Zusammenarbeit fördert. Ein Hauptprinzip ist die Peer-Produktion, bei der Produkte wie Quellcode, Blaupausen und Dokumentationen der Öffentlichkeit frei zugänglich sind}
}

\newglossaryentry{MITLizenz}
{   name={MIT-Lizenz},
    description={Die MIT-Lizenz ist eine Open-Source Lizenz, die das freie Wiederverwenden der unter ihr stehenden Software}
}

\newglossaryentry{GraphQL}
{   name={GraphQL},
    description={GraphQL ist eine Open-Source Datenabfrage und -manipulations Sprache für APIs}
}

\newglossaryentry{Apollo}
{   name={Apollo},
    description={Apollo ist eine Open-Source Platform zum Entwickeln von GraphQL APIs}
}

\newglossaryentry{Git}
{   name={Git},
    description={Git ist eine Software zum Verfolgen von Änderungen in Quellcode, die normalerweise zur Koordinierung der Arbeit zwischen Programmierern verwendet wird}
}

\newglossaryentry{Gitmoji}
{   name={Gitmoji},
    description={Gitmoji ist ein Prinzip nach dem man bestimmte Emojis für bestimmte Code Änderungen verwendet. Dies ermöglicht es schnell den Zweck oder die Absicht eines Commits zu identifizieren}
}

\newglossaryentry{CreateReactApp}
{   name={Create-React-App},
    description={Create-React-App hilft dabei eine Basisumgebung zum Starten einer neuen Einzelseitenanwendung einzurichten}
}

\newglossaryentry{TypeScript}
{   name={TypeScript},
    description={TypeScript ist ein Superset von JavaScript und fügt der Programmiersprache statische Typisierungen hinzu}
}

\newglossaryentry{Webpack}
{   name={Webpack},
    description={Webpack ist ein Bundler der in erster Linie für JavaScript gedacht ist. Er kann jedoch auch Front-End-Assets wie HTML, CSS und Bilder transformieren, wenn die entsprechenden Loader enthalten sind}
}

\newglossaryentry{HotReloading}
{   name={Hot-Reloading},
    description={Die Idee hinter dem Hot-Reloading ist es die App bei Quellcode Änderungen am laufen zu halten und neue Versionen der Dateien einzufügen. Auf diese Weise verliert man nichts vom State, was besonders nützlich ist wenn man die Benutzeroberfläche optimiert}
}

\newglossaryentry{MobX}
{   name={MobX},
    description={MobX ist eine einfache, skalierbare und leistungsstarke JavaScript State-Management Bibliothek}
}

\newglossaryentry{ESLint}
{   name={ESLint},
    description={ESLint ist ein konfigurierbares statisches Code-Analyse Tool zum Identifizieren problematischer Muster in Quellcode}
}

\newglossaryentry{Prettier}
{   name={Prettier},
    description={Prettier ist ein konfigurierbares Code-Formatier Tool}
}

\newglossaryentry{MaterialUI}
{   name={Material-UI},
    description={Material-UI ist eine Front-End React-Komponenten Bibliothek die Google's Material Design implementiert}
}

\newglossaryentry{MaterialDesign}
{   name={Material-Design},
    description={Material-Design ist eine von Google entwickelte Design-Sprache die ein anpassbares System von Richtlinien, Komponenten und Tools darstellt}
}

\newglossaryentry{ReactSpring}
{   name={React-Spring},
    description={React-Spring ist eine Animations-Bibliothek die es erlaubt komplexe CSS-Animationen in JavaScript / React zu erstellen}
}

\newglossaryentry{ReactUseGesture}
{   name={React-Use-Gesture},
    description={React-Use-Gesture ist eine JavaScript-Bibliothek die es erlaubt fortgeschrittene Benutzerinteraktionen mit wenigen Codezeilen zu implementieren}
}

\newglossaryentry{JSS}
{   name={JSS},
    description={JSS ist ein Tool, welches es erlaubt CSS-Regeln in JavaScript konfliktfrei und wiederverwendbar zu beschreiben, und zu manipulieren}
}

\newglossaryentry{ReactHook}
{   name={React-Hook},
    plural={React-Hooks},
    description={React-Hooks sind eine neue Ergänzung in React 16.8. Mit ihnen lassen sich State und andere React Klassen-Komponenten spezifische Features in React Methoden-Komponenten verwenden}
}

\newglossaryentry{ReactHookForm}
{   name={React-Hook-Form},
    description={React-Hook-Form ist eine JavaScript-Bibliothek, mit der man Formulare in React performant und unkompliziert validieren kann}
}

\newglossaryentry{TMDb}
{   name={TMDb},
    description={The Movie Database (TMDb) ist eine beliebte, von Benutzern bearbeitbare Datenbank für Filme und Serien}
}

\newglossaryentry{Proxy}
{   name={Proxy},
    description={Ein Proxy ist eine Serveranwendung, die als Vermittler für Anforderungen von Clients fungiert, die Ressourcen von anderen Servern suchen, die diese Ressourcen bereitstellen}
}

\newglossaryentry{Mongoose}
{   name={Mongoose},
    description={Mongoose ist eine Back-End, JavaScript-Bibliothek die zugleich als MongoDB Treiber dient und es ermöglicht MongoDB-Schemas zu definieren}
}

\newglossaryentry{Bcrypt}
{   name={Bcrypt},
    description={Bcrypt ist eine Passwort-Hashing- / Einwegmethode die auf dem Blowfish-Chiffre basiert. Durch die verwendung eines Salting Schützt sie sich vor Regenbogentabellen-Angriffen}
}

\newglossaryentry{JWT}
{   name={JWT},
    description={JSON-Web-Token (kurz JWT) ist ein Standard zum Erstellen von JSON-Daten mit optionaler Signatur und / oder Verschlüsselung}
}

\newglossaryentry{Transpiler}
{   name={Transpiler},
    description={Ein Transpiler ist eine Art Übersetzer, der den Quellcode eines Programms als Eingabe verwendet und einen äquivalenten Quellcode in einer anderen Programmiersprache erzeugt}
}

\newglossaryentry{Salt}
{   name={Salt},
    description={In der Kryptographie ist ein Salt ein Datensatz, der als zusätzliche Eingabe für Verschlüsselungs-Einwegmethoden verwendet wird}
}

\newglossaryentry{GraphQLCodegen}
{   name={GraphQL-Codegen},
    description={GraphQL-Codegen ist ein Tool zur automatisierten Quellcode generierung basierend auf GraphQL Schemas und Operationen}
}

\newglossaryentry{Docker}
{   name={Docker},
    description={Docker ist ein Open-Source Projekt zur Automatisierung der Bereitstellung von Anwendungen. Es werden mithilfe der Virtualisierung auf Betriebssystemebene Pakete bereitgestellt, die als Container bezeichnet werden}
}

\newglossaryentry{DockerCompose}
{   name={Docker-Compose},
    description={Docker-Compose wird zum Ausführen mehrerer Docker-Container als ein einziger Dienst verwendet}
}
