\newglossaryentry{Front-End}
{   name={Front-End},
    description={Das Front-End bezeichnet den für den Endbenutzer sichtbaren Teil einer Anwendung. (Vergleich Back-End)}
}

\newglossaryentry{Back-End}
{   name={Back-End},
    description={Das Back-End bezeichnet den für den Endbenutzer nicht sichtbaren Teil einer Anwendung. (Vergleich Front-End)}
}

\newglossaryentry{React}
{   name={React},
    description={React ist eine Open-Source, Front-End, JavaScript-Bibliothek zum Erstellen von Benutzeroberflächen}
}

\newglossaryentry{NodeJS}
{   name={NodeJS},
    description={NodeJS ist eine Open-Source, Back-End, JavaScript-Laufzeitumgebung die es ermöglicht JavaScript-Code außerhalb eines Webbrowsers auszuführen}
}

\newglossaryentry{MongoDB}
{   name={MongoDB},
    description={MongoDB ist ein dokumenten-orientiertes Datenbank Programm}
}

\newglossaryentry{NginX}
{   name={NginX},
    description={NginX ist ein Webserver der auch als Reverse-Proxy, Load-Balancer, Mail-Proxy und HTTP cache verwendet werden kann}
}

\newglossaryentry{Express}
{   name={Express},
    description={Express ist eine Open-Source, Back-End, JavaScript-Bibliothek zum Erstellen von Webanwendungen und APIs unter NodeJS}
}

\newglossaryentry{Open-Source}
{   name={Open-Source},
    description={Das Open-Source-Modell ist ein dezentrales Softwareentwicklungsmodell, das eine offene Zusammenarbeit fördert. Ein Hauptprinzip ist die Peer-Produktion, bei der Produkte wie Quellcode, Blaupausen und Dokumentationen der Öffentlichkeit frei zugänglich sind}
}

\newglossaryentry{MIT-Lizenz}
{   name={MIT-Lizenz},
    description={Die MIT-Lizenz ist eine Open-Source Lizenz, die das freie Wiederverwenden der unter ihr stehenden Software}
}

\newglossaryentry{GraphQL}
{   name={GraphQL},
    description={GraphQL ist eine Open-Source Datenabfrage und -manipulations Sprache für APIs}
}

\newglossaryentry{Apollo}
{   name={Apollo},
    description={Apollo ist eine Open-Source Platform zum Entwickeln von GraphQL APIs}
}

\newglossaryentry{Git}
{   name={Git},
    description={Git ist eine Software zum Verfolgen von Änderungen in Quellcode, die normalerweise zur Koordinierung der Arbeit zwischen Programmierern verwendet wird}
}

\newglossaryentry{Gitmoji}
{   name={Gitmoji},
    description={Gitmoji ist ein Prinzip nach dem man bestimmte Emojis für bestimmte Code Änderungen verwendet. Dies ermöglicht es schnell den Zweck oder die Absicht eines Commits zu identifizieren}
}
