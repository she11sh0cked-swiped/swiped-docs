\newglossaryentry{Front-End}
{   name={Front-End},
    description={Das Front-End bezeichnet den für den Endbenutzer sichtbaren Teil einer Anwendung. (Vergleich \gls{Back-End})}
}

\newglossaryentry{Back-End}
{   name={Back-End},
    description={Das Back-End bezeichnet den für den Endbenutzer nicht sichtbaren Teil einer Anwendung. (Vergleich \gls{Front-End})}
}

\newglossaryentry{React}
{   name={React},
    description={React ist eine \gls{Open-Source}, \gls{Front-End}, \gls{JavaScript}-Bibliothek zum Erstellen von Benutzeroberflächen}
}

\newglossaryentry{NodeJS}
{   name={NodeJS},
    description={NodeJS ist eine \gls{Open-Source}, \gls{Back-End}, \gls{JavaScript}-Laufzeitumgebung die es ermöglicht \gls{JavaScript}-Code außerhalb eines Webbrowsers auszuführen}
}

\newglossaryentry{Open-Source}
{   name={Open-Source},
    description={Open-Source ist Quellcode, der für mögliche Änderungen und Weiterverteilungen frei verfügbar gemacht wird}
}

\newglossaryentry{JavaScript}
{   name={JavaScript},
    description={JavaScript ist eine Programmiersprache die oft im Kontext der Webentwicklung verwendet wird}
}

\newglossaryentry{MongoDB}
{   name={MongoDB},
    description={MongoDB ist ein dokumenten-orientiertes Datenbank Programm}
}
