\newglossaryentry{Front-End}
{   name={Front-End},
    description={Das Front-End bezeichnet den für den Endbenutzer sichtbaren Teil einer Anwendung. (Vergleich \gls{Back-End})}
}

\newglossaryentry{Back-End}
{   name={Back-End},
    description={Das Back-End bezeichnet den für den Endbenutzer nicht sichtbaren Teil einer Anwendung. (Vergleich \gls{Front-End})}
}

\newglossaryentry{React}
{   name={React},
    description={React ist eine Open-Source, \gls{Front-End}, JavaScript-Bibliothek zum Erstellen von Benutzeroberflächen}
}

\newglossaryentry{NodeJS}
{   name={NodeJS},
    description={NodeJS ist eine Open-Source, \gls{Back-End}, JavaScript-Laufzeitumgebung die es ermöglicht JavaScript-Code außerhalb eines Webbrowsers auszuführen}
}

\newglossaryentry{MongoDB}
{   name={MongoDB},
    description={MongoDB ist ein dokumenten-orientiertes Datenbank Programm}
}

\newglossaryentry{NginX}
{   name={NginX},
    description={NginX ist ein Webserver der auch als Reverse-Proxy, Load-Balancer, Mail-Proxy und HTTP cache verwendet werden kann}
}

\newglossaryentry{Express}
{   name={Express},
    description={Express ist eine Open-Source, \gls{Back-End}, JavaScript-Bibliothek zum Erstellen von Webanwendungen und APIs unter \gls{NodeJS}}
}
